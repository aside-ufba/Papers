%% Context: 
From 1991 to 2015, 60 papers published in the
ASE and SCAM conferences 
introduced static analysis prototypes or tools
as academic software developed to support research.
%% Objective:
In this study, we characterize such academic software concerning
sustainability.
%% Method: 
We performed an exploratory study regarding publicization (whether the software
is available from an explicitly given URL in a software
publication), evolution stage (initial development, evolution, servicing,
phase-out or close-down), and recognition (the way others mention the software
in their papers). Thereby, we discussed the results under the umbrella of
software sustainability.
%% Results:
Results 
for the academic software for static analysis published at ASE and SCAM, 
show that
40\%    %of the academic software for static analysis 
are not publicly available from the URL informed by the authors;
78\%    %of the academic software for static analysis 
are in an initial development stage, discontinued, or closed-down; 
23\% 
has no mentions in relevant digital libraries
besides the original software publication, but
30\% 
received contributions to their source code. 
We observed that a large number of academic static analysis software
has inadequate publicization, short life cycles and low recognition.
%% Conclusion:
A systematic analysis of publicization, software life cycle, and recognition of
academic software is viable, and its results may be useful to support
rapid decision-making on adopting academic software for use 
or even as a target for contribution.
The results may also promote a more inclusive  view of scientific reputation
with respect to the academic software produced by researchers.
