\section{Conclusions} \label{sec:conclusions}

The development of academic software, in a sustainable way, opens doors to
raising the overall quality of scientific research, promoting visibility,
reproducibility, and fostering a collaborative environment in opposition to the
traditional model of competition that permeates the system of scientific
reputation and credit.
%
In software engineering, especially in the field of static analysis, with
tradition in the development of tools to support research in different areas of
computer science, the concern with the technical sustainability of academic
software cannot be disregarded.

This work characterized academic software for static software analysis
published until 2015 in scientific papers from the ASE and SCAM conferences,
concerning their technical sustainability, defined regarding publicization,
stage in software life cycle, and recognition by scientific peers.

Academic software projects are not yet recognized by Science as first-class
citizens, even though these projects are indispensable for understanding the
studies in which they were developed. In this study, we showed that a
significant amount (40\%) of a set of academic software projects minimally
identified by their authors with name and URL is currently unavailable.

As a result of the unavailability of academic software, 
data may be lost, research can take longer than necessary, 
and researchers may be unable to reach the efficiency
they could have when working with academic software.

\myparagraph{Future Work}
To assess why some tools in closedown stage continue to be cited along the time should be a topic for future investigation, once that the tools are not available anymore, maybe these tools continue to receive attention from scientists due to the scientific relevance of their techniques and algorithms.
%
Another future step in this study is to investigate whether the results found regarding academic software recognition are similar to general software engineering publications.

%For instance, what percentage of purely scientific papers published in ASE and
%SCAM in the same period has no recognition (i.e., no citations)?
