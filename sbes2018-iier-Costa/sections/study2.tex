\section{Life cycle of Academic Software} 
\label{study3}

The academic software projects featured in this study do not have typical life
cycles of proprietary commercial software products.  Therefore, they are
considered as free/libre/open source software projects and characterized with
the life cycle model proposed by \cite{capiluppi2007adapting},
adapted from \cite{rajlich2000staged}.

We used the number of modules\footnote{the term \textit{module}
refers to a unit that makes up a software system, such as ``classes'',
``functions'' or ``source files''.} to define the evolution stage of academic
software, and calculated it by using static analysis of source code.  Studies
show that the number of modules in software has a tendency to stabilize over
time and thus this measure can be used to characterize their stage of evolution
\cite{capiluppi2007adapting}.

We also take into account the types to classify academic software proposed by
Howison \cite{howison2011scientific}. An academic software can be developed in
a model of {\it software as supporting service} living outside the reputation
economy of science or as a {\it software for academic credit}, having a strong
association with the science ecosystem. The second type considers an academic
software as {\it incidental software} written purely to facilitate research, or
as {\it a parallel software practice} related to a situation facing many
scientists who release software for others to use, or yet as {\it a software
subfield} where publications about software are the primary contributions.

\myparagraph{Scoping}
%
The goal of the study is to analyze the \textit{published static analysis
software projects} with the purpose of \textit{characterizing them}, concerning
their \textit{evolution stage}, from the perspective of \textit{end-user
scientists and software developers}, in the context of their software
ecosystems.

In this study, we investigated the following research question regarding
academic software for static analysis published at ASE and SCAM
as well as its evolution stage of the life cycle:

\newcommand{\StudyThreeQuestionOne}{
	\textbf{(RQ2.1)} \textit{In which evolution stage of the software life cycle defined by
Capiluppi's model \cite{capiluppi2007adapting}, are these academic software?}
}

\noindent \StudyThreeQuestionOne We investigate  which academic software
projects of static analysis continue to evolve after its initial publication and 
in which evolution stage they are.

Measurements required in this study include:
  (1) the number of releases of each project with information about its version and date;
  (2) the number of releases with available source code for download; and
  (3) the number of modules in the source code of each version/release.

\myparagraph{Data Collection}
%
We collected data from the academic software project website, related
publications and software repositories; we surveyed neither developers nor
researchers. We present these data in Table \ref{life-cycle-table}.

\begin{table}[ht]
\scalefont{0.9}
  \caption{Number of projects per software life cycle stage}
  \centering
  \begin{tabular}{l c c}
    \hline
    {\bf Stage} & {\bf Academic Software} & {\bf \%} \\
    \hline
      Initial development & 20 & 33\% \\
      Evolution           & 2  & 3\%  \\
      Servicing           & 6  & 10\% \\
      Phaseout            & 3  & 5\%  \\
      Closedown           & 24 & 40\% \\
      Unknown             & 5  & 8\%  \\
    \hline
      {\bf Total}         & 60 & 100\% \\
    \hline
    \label{life-cycle-table}
  \end{tabular}
\scalefont{1}
\end{table}


We manually inspected the website of all projects looking for data about releases and versions, we manually downloaded each project release with source code available, and then we run the Analizo\footnote{\url{http://www.analizo.org}} on each project and each release to collect the number of modules. We just analyzed projects written in C, C++, C\#, and Java. We wrote a script\footnote{\url{https://github.com/joenio/dissertacao-ufba-2016/blob/master/bin/run-analizo}} to automate the Analizo execution on each project ant their releases.

\myparagraph{Results and Analysis}
%
The implication on this study is:

\paragraph{\bf Evolution stage of academic software}
Regarding question \textbf{RQ2.1}, most of the \SoftwareCount \ academic
software analyzed, 44 projects (73\%), are in initial development or closedown
stage.  From the remaining 16 projects (27\%), 11 projects (22\%) are in
evolution, servicing or phaseout stages, and for five projects (8\%) the stage
could not be determined, as detailed in Table \ref{life-cycle-table}.
%
Academic software in \textit{initial development} (a first functioning version
is released) share characteristics with incidental software
\cite{howison2011scientific} (developed mainly to support research).  Academic
software in \textit{evolution} (capabilities and functionality are extended to
meet user needs) or \textit{servicing} (minor defect repairs and simple
functional changes) share characteristics  with {\it parallel software
practice} (developed and evolved to be used by other researchers). Besides, it
as {\it software sub-field} (the software itself is considered a primary
contribution to science).

\myparagraph{Threats to validity}
%
We performed the analysis and the interpretation of results based only on the
data collected from the academic software project website, related publications
and software repositories; we surveyed neither developers nor researchers. This
scenario can harm validity in case of outdated or incomplete documentation.
However, our study was exploratory and aimed to draw a general picture of the
domain of static analysis and not necessarily bring evidence on each of the
projects individually.
