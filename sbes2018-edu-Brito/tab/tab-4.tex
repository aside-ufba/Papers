\begin{table*}
	\centering
	\caption {Learning Approaches - Facet 3}  
		{\begin{tabular}{ p{1.5in} | p{3.7in} }
			\bf Category & \bf Description \\
			\hline

			\bf Active learning (general) & General term that refers to several models of education that focus the responsibility of learning on learners. Usually students engage in higher-order thinking tasks such as analysis, synthesis, and evaluation. \\
			\bf Case-based learning & Approach where students develop skills in analytical thinking and reflective judgment by reading and discussing complex, real-life scenarios. A case is already organized and synthesized for students. \\
			\bf Game-based learning & Learning that involves students in some sort of competition or achievement in relationship to an educational goal. Attempts to increase student motivation by providing a playful environment. \\
			\bf Peer/Group/Team learning & Educational practices in which students interact with other students to attain educational goals. Such approaches enhance the value of interaction and information sharing among peers. \\
			\bf Project-/Problem-/Inquiry-based learning & A collection of approaches that use projects or problems to drive the learning process. Students learn about a subject through the experience of problem solving, by working in groups with the help of facilitators. Assessment is performance-based and authentic. \\
			\bf Studio-based learning & Approach from professional education, where students undertake a project under the supervision of a master designer. It uses a learning cycle of construction, presentation, critique and response, that is repeated until project completion. \\
			\bf Project method &  The traditional method where students participate in a project development. \\
			\bf Other & Other approaches different from the previous categories.\\
			\bf Not specified & Authors do not state the learning approach used. \\
			\bf Does not apply & Paper is not related to an experience where a learning approach is needed. \\
		\end{tabular}}
		\label{tab:activeLearning}
\end{table*}
