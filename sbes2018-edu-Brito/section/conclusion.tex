% Conclusion

We presented an update to a systematic mapping study
performed by our research group in 2013 to investigate
the use of FLOSS projects in Software Engineering Education. 
%
The first mapping selected 72 primary studies,
published from 1998 to 2012, and
categorized them according to nine facets 
(software engineering area, research type, learning approach,
assessment perspective, assessment type, 
approach goal, curriculum choice, control level and project choice).
The updated mapping selected 33 primary studies
published from  2013 to 2017, and
categorized them according to the same nine facets.
%
Moreover, we compared the results to provide
a comprehensive overview on how FLOSS projects have been used in the context of SEE in the last 20 years.

There has been a steady but yet small body of research 
that addresses the pedagogical use of FLOSS projects in SEE; 
it includes 105 primary studies, developed by a small number of
groups and researchers, and mostly published in conferences 
on computer science education.
% with a growing number of publications and researchers interested in the subject.
From 105 studies in 20 years, 
only one was classified as Experiment/Quasi-experiment in
the Research Type facet.
Despite the increase in the number of experience reports,
evidence shows that the research area is not mature yet.

%
The updated SMS is a valuable asset both 
to researchers interested in the identified trends and gaps, and 
to instructors interested in trying out their own experiences in their classes. 



