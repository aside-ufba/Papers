% abstract

\textbf{Context:}
Free/Libre/Open Source Software (FLOSS) projects
have been used in Software Engineering Education (SEE)
to address the need for more realistic settings
that reduce the gap between
software engineering (SE) courses and industry needs.
A systematic mapping study (SMS) performed in 2013
structured the research area on the use of FLOSS projects in SEE.
%
\textbf{Objective:} 
Update the 2013 SMS with studies published in the last five years,
classifying and summarizing them
to discuss trends and identify research gaps
in the context of the use of FLOSS projects in SEE.
%
\textbf{Method:} 
We retrieved and analyzed a set of 4132 papers 
published from 2013 to 2017,
from which 33 papers were selected and classified.
%according to the scheme proposed in the original SMS.
We analyzed the new results and compared them
with those from the previous SMS
to confirm or discover trends.
%
\textbf{Results:} 
The updated mapping summarizes
the studies published in the last five years,
most of them in conferences.
Our analysis confirmed trends previously observed for three facets
(SE area, curriculum choice and assessment type) and discovered new trends for other facets.
%
\textbf{Conclusion:}
Studies report the use of FLOSS projects
in regular, comprehensive SE courses.
The prevalence of experience reports
over solution proposals %as a research type
in the last five years may indicate that
researchers are more concerned with the use and evaluation
of existing proposals, although there are still opportunities
for more empirical work based on 
sound educational research methods.

