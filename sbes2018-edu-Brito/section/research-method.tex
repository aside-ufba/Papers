% \section{The Systematic Mapping Study Update Process} \label{sec:sms:process}
% The update process

Figure~\ref{fig_smsProcess3} presents the process 
that guided our update to the SMS conducted by~\citet{2015:CSE:nascimento}.
%Steps marked with \verb1*1 were reused from the previous SMS. 
%Their related outcomes (e.g., review scope and classification scheme),
%were the same as defined in~\cite{2015:CSE:nascimento}.
%
We used the StArt (State of the Art through Systematic Review) tool\footnote{http://lapes.dc.ufscar.br/tools/start\_tool}
and spreadsheets to manage the
data generated during the update process.

%\paragraph{Research Questions}  
The three research questions defined by
~\citet{2015:CSE:nascimento} were also reused in this update:
both the main question and the two secondary questions.
In this paper, we focus on the main research question:

\begin{description}
\item [RQ1.] How are Open Source Projects used in Software Engineering Education? 
\end{description}

%\begin{description}
%\item [RQ2.] Which of the initiatives combine open source projects with active learning in software engineering courses?
%\end{description}
%\begin{description}
%\item [RQ3.] How is student learning assessed in the reviewed initiatives?
%\end{description}

The review scope included the same search strings 
and inclusion/exclusion criteria defined by
~\citeauthor{2015:CSE:nascimento} 
for paper searching and screening.
Detailed information on the SMS protocol
can be found at the original SMS website\footnote{https://sites.google.com/site/dmcnascimento/mapping} 
and detailed information on the updated SMS protocol is available as supplementary material\footnote{https://github.com/Moara/mapping/blob/master/Disponibilizar/Protocolo.pdf}.
% original SMS \cite{2015:CSE:nascimento}.

\subsection{Search}
We conducted the search from March 22 to April 28, 2018. 
The search strategy included 
the same scientific electronic databases
used by~\citet{2015:CSE:nascimento}:
Engineering Village\footnote{http://www.engineeringvillage.com},
IEEE Xplore\footnote{http://ieeexplore.ieee.org}, 
ACM\footnote{http://portal.acm.org}, 
Scopus\footnote{http://www.info.sciverse.com/scopus}, 
Springer\footnote{http://www.springer.com} and 
Science Direct\footnote{http://www.elsevier.com}.
We retrieved 18,056 papers, 
from which 13,924 were duplicates 
(various studies were indexed by more than one digital library), 
leading to 4,132 papers selected for screening.

\subsection{Screening}
Four inclusion criteria (IC) and 20 exclusion criteria (EC) 
defined by~\citet{2015:CSE:nascimento} were reused. 
For instance: 
(IC.1)~Studies that address the use of FLOSS projects to learn/teach Software Engineering should be included, regardless of their application in either SE programs or in other programs;
%(e.g. Computer Science, Computer Engineering, Information Systems, Information Technology or Informatics);
(EC.1)~Documents written in languages other than English should be excluded; (EC.2)~Documents whose full text is not available should be excluded; and (EC.3)~Studies whose main content is not related to learning or teaching Software Engineering should be excluded.

We applied two filters during screening.
First, the two first authors individually examined title and abstract
of each primary study, and marked
each study as either included or excluded. 
A third reviewer compared the results,  
analyzed conflicts and took the final decision. 
In this first step, 39 studies were included.

Then, the two first authors reexamined 
the remaining studies, 
skimming through introduction and conclusion. 
A third reviewer compared their results, 
and again, took the final decision, 
resulting in 33 primary studies selected 
for classification.

%MMMM studies. 
%Finally,  we followed a ``snow-balling'' procedure~\cite{Budgen} that supported us in the identification of other PPPP relevant studies.
%Todos os artigos foram baseadas na string de busca, não fizemos snow-balling.
%A total of 33 primary studies were selected for classification.

\begin{table*}[tb]
%\begin{table}
	\centering
	\caption {Defined Facets~\cite{2015:CSE:nascimento}}  
		{\begin{tabular}{c|l|p{5in}} 
			 & \bf Facet & \bf Description \\
			\hline
			\bf 1 	& \small \bf Software Engineering Area 
            		& The SE topic(s) addressed in the study 
(software engineering in general, requirements, models and methods, design and architecture, quality, testing, evolution and maintenance, development and construction, process, management and configuration management), based on the SWEBOK \citeyearpar{swebok}. \\
			\bf 2 & \small \bf Research Type & The research approach used in the paper 
            	(experience report, case study, action research, experiment/quasi-experiment, survey, opinion paper, solution proposal, philosophical paper
            -- adapted from \citet{Petersen}). \\
%Table~\ref{tab:researchTypeStudies} provides a description of each category,  \\
%			\bf 3 & Learning Approach & The pedagogical approach that was applied together with OSP in SE courses. Table~\ref{tab:activeLearning} presents a description of each category. \\
%			\bf 4 & Assessment Perspective & The perspective from which student learning is evaluated -- see Table~\ref{tab:assessmentPerspective}. \\
			\bf 3 & \small \bf Learning Approach & The pedagogical approach 
            that was applied together with FLOSS projects in SE courses 
            -- not presented in this paper. \\
			\bf 4 & \small \bf Assessment Perspective & The perspective from which
            student learning is evaluated -- not presented in this paper. \\
			\bf 5 & \small \bf Assessment Type & The instrument used to assess student
            learning -- see Table~\ref{tab:assessmentType}. \\
			\bf 6 & \small \bf Approach Goal & Which goal in introducing FLOSS projects
            in SE courses was prevalent (either learning SE principles/concepts 
            or learning to develop FLOSS). \\
			\bf 7 & \small \bf Curriculum Choice & How the content was embodied in
            curriculum -- see Table~\ref{tab:curriculumApproach}. \\
			\bf 8 & \small \bf Control Level & How much control faculty/staff 
            had on the FLOSS project -- see Table~\ref{tab:controlLevel}. \\
			\bf 9 & \small \bf Project Choice & How the FLOSS projects were chosen  -- see Table~\ref{tab:projectChoice}. \\		
		\end{tabular}}
	\label{tab:facets}
%\end{table}
\end{table*}


%\paragraph{Keywording Relevant Topics}
\subsection{Classification}
The SMS update reused the classification scheme 
proposed by~\citet{2015:CSE:nascimento}.
Table~\ref{tab:facets} presents the nine facets defined for classification purposes.
Additional information about the facets is provided in Appendix~\ref{sec:facets}.
%Tables~\ref{tab:assessmentType}, \ref{tab:curriculumApproach}, \ref{tab:controlLevel}, and \ref{tab:projectChoice}.

\subsection{Data extraction}
Each primary study accepted in the screening process 
was fully read to collect the required data. 
We extracted information on title, authors, 
authors' affiliation details, 
venue and year of publication for each selected paper, 
and additional information required to classify it 
according to each facet. 
During data extraction, we also looked for information 
to characterize long-term projects and 
the continuity of research work.

